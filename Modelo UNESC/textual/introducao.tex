% ----------------------------------------------------------
% Introdução
%Ex: \chapter{TÍTULO A SER IMPRESSO NO CORPO DO TEXTO}{Título no cabeçalho}{Título no Sumario}
% ----------------------------------------------------------
\chapter{INTRODUÇÃO} % se usar o \chapter* ele não vai colocar no sumario
%\addcontentsline{toc}{chapter}{Introdução} inclui manualmente no sumario sem numeração
Delimita o assunto, define brevemente os objetivos do trabalho e as razões de sua elaboração, bem como as relações existentes com outros trabalhos. Apresenta o problema e as questões norteadoras ou hipóteses. Não deve antecipar conclusões e recomendações.

\section{APRESENTAÇÃO}
Use este espaço para apresentar o tema e problema para seu leitor.

\section{DESCRIÇÃO DO PROBLEMA}
Descreva aqui seu problema de pesquisa.

\section{JUSTIFICATIVA}
Escreva aqui sua justificativa.

\section{OBJETIVOS}

\subsection{Objetivo geral}
Descreva aqui o objetivo geral a ser abordado pelo estudo apresentado.

\subsection{Objetivos específicos}
Desdobramento do objetivo geral. 
 \begin{itemize}
	\item Item 1
	\item Item 2
	\item Item 3
\end{itemize}

 \section{METODOLOGIA}
 
 Use este espaço para detalhar os procedimentos métodologico adotado.
