% resumo em português
\begin{resumo}
\noindent
Consiste na apresentação dos pontos relevantes de um texto. O resumo deve dar uma visão rápida e clara do trabalho; constitui-se em uma sequência de frases concisas e objetivas e não de uma simples enumeração de tópicos. Apresenta os objetivos do estudo, o problema, a metodologia, resultados alcançados e conclusão. Deve ser digitado em espaço simples e sem parágrafos, não ultrapassando a 500 palavras.
 %Aqui vai o resumo

 \vspace{0.2cm}

 
 \textbf{Palavras-chave:} Escrever de três a cinco palavras representativas do conteúdo do trabalho, separadas entre si por ponto e finalizadas também por ponto.
\end{resumo}

% resumo em inglês
\begin{resumo}[Abstract]	
 	\begin{otherlanguage*}{english}
 	\noindent 

	Resumo na língua inglesa. % Aqui vai o abstract

   \vspace{0.2cm}
 
   
    \textbf{Key-words:} Latex. Abntex. Text editoration.	
 	\end{otherlanguage*}
\end{resumo}